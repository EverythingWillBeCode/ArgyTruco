\documentclass[12pt,a4paper]{article}
\usepackage[spanish]{babel}
% \usepackage[utf8]{inputenc}
% \usepackage[dvips]{graphicx}
\usepackage{listings}


\newcommand{\HRule}{\rule{\linewidth}{1mm}}

%\lstset{fancyvrb=true}
\lstset{
	basicstyle=\small,
	tabsize=2,
	numbers=left,
	captionpos=b		
}


\title{Inteligencia Artificial\\ 
\large Informe Final
}
\author{Botta Mariano \\ Campostrini Esteban}
\date{ \small 13 de Septiembre del 2013}

\begin{document}

\maketitle 

\section{Introducci\'on}
El proyecto consiste en diseñar y desarrollar una inteligencia artificial capaz de jugar al juego de cartas conocido como Truco.
Una descripci\'on completa del juego puede verse en \cite{reglas}. El dise\~no del juego fue pensado para permitir partidas 
entre un humano y una maquina. No se tuvieron en cuenta todas las caracter\'isticas del juego, por ejemplo el canto de "Flor" 
no esta soportado. 

Luego de un an\'alisis del juego optamos por representar la inteligencia artificial mediante un sistema basado en conocimientos, 
más precisamente un sistema experto con reglas de producción de tipo deductivo (Forward Chaining). Si bien no se utiliz\'o un motor
deductivo en particular, creemos que los pasos seguidos por el sistema a la hora de tomar las desiciones representan el ciclo
base de un motor deductivo.


\section{Lenguaje y Desarrollo}
El juego junto con la inteligencia fueron desarrollados desde cero en una interfaz web. 

La interfaz gr\'afica esta hecha con HTML y CSS. La l\'ogica del juego est\'a escrita en Javascript y est\'a separada en 2 modulos principales.  
\begin{itemize}
\item El modulo \textbf{truco.js} provee las estructuras de datos necesarias para la representacion de la informacion utilizada durante el juego,
controla la interacci\'on entre el jugador y la m\'aquina, es decir, determina quien puede jugar en un momento determinado y lleva cuenta del
estado global del juego.

\item El modulo \textbf{ia.js} continene la inteligencia artificial de la m\'aquina. Aqui se encuentra codificado el razonamiento realizado por la maquina
a la hora de tomar una desici\'on.

\end{itemize}

A modo de dar una visi\'on general, el sistema funciona de la siguiente manera: \\
Si el usuario es mano, cualquier acci\'ion que este haga ser\'a procesada por el modulo \textbf{truco.js}. Este se encargar\'a de contextualizar la
acci\'on (por ejemplo, determinar en que mano se realiz\'o el canto y/o si hubo un canto anterior a ese) y llamar\'a al correspondiente m\'etodo del
m\'odulo \textbf{ia.js} que usando esta informaci\'on como conocimiento de dominio, determinar\'a con qu\'e acci\'on responder.



\section{Conocimiento de Dominio}

\subsection{Sobre el enga\~no en el truco}

El truco es un tipo de juego con informaci\'on parcial en el sentido en que al menos uno de los jugadores
conoce solo un estado parcial del juego (en caso de un desarrollo "normal" del juego, esto deber\'ia 
pasar para todos los jugadores involucrados). Esto implica una complejidad mayor a la hora de tomar desiciones
ya que es necesario tener en cuenta las jugadas pasadas como las posibles jugadas. 
El hecho de que los jugadores no conozcan exactamente el estado del resto de los jugadores puede ser aprovechado
de manera beneficiosa para intentar inducir estados falsos que den lugar a malas desiciones de juego 
por parte de otro de los jugadores. En el truco este enga\~no puede manifestarse de dos formas (que osamos en llamar
activa y pasiva)\\
\textbf{Activa:} el jugador toma la iniciativa mediante alg\'un canto (envido o truco) con el fin de hacerle creer al oponente que tiene
cartas o puntos que en realidad no tiene.\\
\textbf{Pasiva:} el jugador plantea un escenario (generalmente mediante el orden en el que juega las cartas) de manera tal de que 
la hipotetizaci\'on hecha por el oponente sobre el estado sea err\'onea llevandolo a tomar acciones 'activas' poco favorables para 
este (basicamente que el jugador que esta intentando ser enga\~nado piense que el otro jugador tiene mas o menos puntos o cartas de 
mayor o menor valor que en las que realidad tiene).\\
Cabe destacar que un intento de enga\~no tanto activo como pasivo esta influenciado tambien por la informaci\'on que uno tiene del otro jugador,
que en la mayoria de los casos sigue siendo parcial,  por lo cual hay una especie de ciclo de mentiras.\\
De aqu\'i, que el enga\~no juega un rol importante en el truco.

\subsection{Funciones principales}
Como se explica en la secci\'on anterior, el enga\~no es parte del juego y este est\'a generalmente influenciado por
la informaci\'on disponible, lo cual tambi\'en implica que no hay una estrategia ganadora. 
A continuaci\'on haremos una descripci\'on de las funciones principales de la IA comentando c\'omo fue tenida en cuenta la 
informaci\'on disponible y de que manera se intento darle "picard\'ia" a la m\'aquina.

%Este juego es muy particular, no existe una f\'ormula ganadora. Depende mucho de la picard\'ia del jugador. Por eso, apesar de estar basado en un sistema experto
%con reglas, en muchos cosas como \'ultima opci\'on se agrego un poco de no-determinismo para crear una sensaci\'on  de que no es una maquin\'a est\'atica. 
%Este no determinismo esta fuertemente influenciado por el conocimiento de dominio.

Luego de analizar la metodolog\'ia de juego, se llego a una divisi\'on en 3 partes diferentes:
\begin{itemize}
\item Truco: se encarga de decidir los cantos del truco
\item Envido: decide el canto del envido
\item Jugar Carta: decide que carta jugar
\end{itemize}
 

\subsubsection{Truco}
%La informaci\'on que consideramos pertinente a la hora de decidir el canto del truco esta
%representada por las siguientes variables:
Esta es la informaci\'on que consideramos pertinente a la hora de decidir el canto del truco:
\begin{itemize}
\item Informaci\'on sobre si el humano cant\'o algo y la m\'aquina debe contestar o si esta puede cantar (variable \textbf{resp})
%\item resp: valor booleano que indica si el humano cantó algo y por lo tanto la m\'aquina tiene que contestar o si esta puede cantar.
\item El \'ultimo canto hasta el momento en la ronda actual (variable \textbf{ultimo})
\item El n\'umero de la mano actual (primera, segunda o tercera)
\item Resultado de las manos anteriores
\item Cartas jugadas en la mano actual (propias o del oponente)
\item Clasificaci\'on de las cartas en mano (cantidad de cartas altas, medias o bajas que tengo).
\item Posibles cartas del oponente seg\'un los puntos cantados en el envido y las cartas que jug\'o.
\end{itemize}

Mediante todos estos atributos se conforman las reglas de inferencia que se dividen (a grandes rasgos) de la 
siguiente forma:

\begin{itemize}
\item 1) Contestar un canto o Cantar
\item 2) Se determina cual es la mano actual
\item 3) Se determina el resultado de la mano anterior (para las manos 2 y 3)
\item 4) Se clasifican las cartas que poseo
\item 5) En función de la situación en la que se encuentre y teniendo en cuenta los datos correspondientes al dominio
	   se decide que cantar (Si, No, Re-Truco, Vale 4)
\end{itemize}

De manera m\'as ilustrativa veamos un extracto del c\'odigo de la funci\'on truco
\begin{figure}[h]
\lstset{language=java,caption=Extracto de la funci\'on truco,label=lst:nicecode}
\begin{lstlisting}

if (resp) {  // Me cantaron, tengo que responder
	switch(nroMano){
		case 0:
			var ran = getRandomInt(0,100);
			switch(ultimo){
		case 'T':
			if (e2.jugador.puntosGanadosEnvido < 2 && 
					(mediaalta) >= 2 && clasif.alta >= 1)
				return 'S';
			if (clasif.baja === 3 && ran <= 50) return 'RT'; 
			if (mediaalta >= 2 && diff < 0  ) return 'RT';  // Te trato de correr
			if (mediaalta >= 1 && diff > 0  ) return 'S';
			return ( ran < 66 ? 'N' : 'S' ); //  (*) Rompe con la reglas staticas
		case 'RT':
		case 'V':
			if (clasif.alta >= 2) return 'S';
			if ((mediaalta) >= 2  && diff > 3 ) return 'S';
			if (e2.jugador.puntosGanadosEnvido >= 2) return 'N';
			return 'N';
			break;
		}
	}
}
\end{lstlisting}
\end{figure}


\noindent En este caso vemos qu\'e sucede cuando el oponente canta algo (la variable \textbf{resp} toma el valor true). Primero determinamos
en que n\'umero de mano nos encontramos. Luego en funci\'on de lo que fue cantado por el oponente (truco - T, re-truco - RT, o vale 4 - V)
y en funci\'on de las cartas en mano as\'i como de los puntos obtenidos en el envido se decide que acci\'on tomar. Se puede ver tambi\'en
una componente aleatorio. Si bien no es un recurso ideal, nos encontramos con el hecho de que una vez colectada toda la informaci\'on posible
(que es incompleta) siempre hay un factor no determinista (simpre hay m\'as de un camino a seguir)
que podr\'ia ser ignorado poniendo reglas fijas pero esto har\'ia que el comportamiento de la m\'aquina sea bastante "r\'igido".

\subsubsection{Envido}
%El conocimiento de dominio que utiliza comprende:
La informaci\'on tenida en cuenta comprende:
\begin{itemize}
\item El \'ultimo canto realizado (variable \textbf{ultimo})
\item La cantidad de puntos en juego (variable \textbf{acumulado})
\item Carta jugada por el oponente (si es que existe) y deducci\'on de posibles puntos (variable \textbf{ultimaCarta})
\item Puntos de Envido en mano (variable \textbf{puntos})
\item Puntos restantes para terminar el partido
\item Puntos con lo que el oponente canto en otras manos
\item Puntos que se pierden si no se acepta el envido (variable \textbf{puntosNoQuerido})
\end{itemize}


Para decidir el canto del envidio, primero se analiza los puntos restantes para terminar la partida debido a que la Falta Envido es diferente al 
resto de los cantos. Si faltan dos o menos puntos para ganar, se responde a cualquier canto de envido cantando la Falta. De igual manera, dadas
las mismas condiciones si no se cant\'o nada y es el turno de la m\'aquina, se canta la Falta.\\
\noindent En caso de estar a mediados del juego se tiene en cuenta si hubo cantos previos, si el oponente ya jug\'o una carta y la cantidad de puntos que tengo en mano. 
Todos estos valores generan una decisi\'on de juego, a la que en \'ultima instancia se le agrega un factor aleatorio, con el fin de evitar nuevamente
un comportamiento fijo en la m\'aquina.

\newpage


\begin{figure}[h]
\lstset{language=java,caption=Extracto de la funci\'on envido,label=lst:nicecode}
\begin{lstlisting}
switch(ultimo){
	case 'E':
		var pRE = this.prob.promedioPuntos(this.envidoS)  ;
		pRE =  pRE === null ? 0 :  pRE - 15 ; 
										                  
		if (ran + posible + diff + acumulado + pRE  <  valor  * 100 ) {
			if (puntos >= 30 ) return  'EE' ; 
			else return 'S';
		} 
		else { 
			if (puntos >= 30 ) return 'S'; 
			else  return 'N';  
		}
		break;
}

\end{lstlisting}
\end{figure}

\noindent En el extracto anterior se puede ver como se procesa el caso en el que el humano canta Envido.
La variable \textbf{pRE} se inicializa con el promedio del historial de puntos con los que el humano
cant\'o envido hasta ese momento. En la siguiente linea	se hace un corrimiento, centrando en cero el valor
(tomando 15 como la mitad de los puntos que uno puede tener en el envido). La idea es que si el humano
solia cantar hasta ese momento con menos de 15 puntos, el valor de pRE va a ser negativo y va a contribuir
a que el primer if evalue a verdadero, dando lugar a que la m\'aquina revire (en terminos m\'as familiares,
si el oponente suele cantar con pocos puntos, despu\'es de un tiempo va a perder algo de credibilidad y voy 
a considerar revirarle en alg\'un momento).\\
\noindent En el If siguiente se puede ver la expresi\'on: 
\begin{center}
	$ ran + posible + diff + acumulado + pRE  <  valor  * 100 $
\end{center}
que representa el recurso que usamos para introducir no determinismo. La idea es tener en cuenta a la hora de revirar
o aceptar el envido, las variables que uno analizar\'ia en un juego normal, ante otras personas. B\'asicamente son 
(est\'an tambi\'en explicadas m\'as arriba): qu\'e es lo que puede tener el oponente, cu\'al es la diferencia de puntos
en el partido entre el opoente y yo, cuantos puntos hay en juego en este momento y el historial de canto del oponente.\\
\noindent A esto decidimos agregarle la componente aleatoria y lo que hacemos es compararlo con un valor que asignamos a
los puntos que tenemos en mano. Los valores a ambos lados de la desigualdad est\'an entre 0 y 100.
%% Revisar estooo ultimo!!!! 



\subsubsection{Jugar Carta}
El conocimiento de dominio que utiliza comprende:
\begin{itemize}
\item Juego primero o  el oponente ya jugo
\item Clasificaci\'on de las cartas
\item Numero de Mano
\item Carta jugada por el oponente
\end{itemize}

Para determinar que carta jugar primero se tiene en cuenta si el oponente jugo. Si es as\'i elige la carta en 
funci\'on de la que esta en la mesa. Depende de si le puede ganar o no.
Si la m\'aquina es mano, por lo general opta por la siguiente f\'ormula: Ganar primera, dejar pasar segunda y ganar tercera. 
Por supuesto hay reglas , que capturan situaciones especiales, provocando una variaci\'on al juego de la m\'aquina. Por ejemplo, si es mano y posee dos cartas altas.


El conocimiento de dominio necesario fue aportado por los integrantes del grupo 

%%\section{Futuro Trabajo}
%% Esto que estoy poniendo es para justificar los random pero hay que chamuyarlo mas
%%Este juego es muy particular, no existe una f\'ormula ganadora. Depende mucho de la picard\'ia del jugador. Por eso, apesar de estar basado en un sistema experto
%%con reglas, en muchos cosas como \'ultima opci\'on se agrego un poco de no-determinismo para crear una sensaci\'on  de que no es una maquin\'a est\'atica. 
%%Esto podr\'ia cambiarse agregando diferentes estilos de juegos, como el mentiroso, el defensivo o el agresivo.   

\begin{thebibliography}{9}

	\bibitem{reglas} \emph{Descripci\'on y Reglas del Truco }. 
	http://es.wikipedia.org/wiki/Truco\_argentino .

\end{thebibliography}
\end{document}

