\documentclass[12pt,a4paper]{article}
\usepackage[spanish]{babel}
\usepackage[utf8]{inputenc}
\usepackage[dvips]{graphicx}
\usepackage{listings}


\newcommand{\HRule}{\rule{\linewidth}{1mm}}


\title{Inteligencia Artificial\\
\\ 
\large Informe Final
}
\author{Botta Mariano \\ Campostrini Esteban}
\date{ \small 13 de Septiembre del 2013}

\begin{document}
\maketitle 
\section{Introducci\'on}
El proyecto consiste en diseñar y desarrollar una inteligencia artificial capaz de jugar al juego de cartas conocido como Truco.
Una descripci\'on completa del juego puede verse en \cite{reglas}. Aunque no se tuvo en cuenta todas las caracter\'isticas del jugo. 
Por ejemplo, el canto de "Flor" no esta soportado. 

Luego de un an\'alisis del juego, decidimos realizar la inteligencia como un sistema basado en conocimientos, más precisamente un sistema experto con reglas de 
producción de tipo deductivo (Forward Chaining)

\section{Lenguaje y Desarrollo}
El juego junto con la inteligencia fueron desarrollados desde cero en una interfaz web. 

La interfaz gr\'afica esta hecha con HTML y CSS. La l\'ogica del juego est\'a escrita en Javascript y est\'a separada en 2 modulos principales.  
\begin{itemize}
\item El modulo truco.js provee las estructuras para la representación de los datos necesarios del juego asi como la interacci\'on entre los dos jugadores. 
En este caso el humano contra la computadora. 

\item El modulo ia.js provee la inteligencia artificial del jugador. Aqu\'i se encuentran ocultas todas las desiciones del jugador.
Durante el juego, hay tres situaciones bien marcadas: 
\begin{itemize}
\item Truco: se encarga de decidir los cantos del truco
\item Envido: decide el canto del envido
\item Jugar Carta: decide que carta jugar
\end{itemize}
\end{itemize}

\section{Conocimiento de Dominio}
Luego de analizar la metodología de juego, se llego a una división en 3 partes diferentes:
El canto del envido
El canto del truco
Selección de carta a jugar
 

Una mirada sobre la función truco
El conocimiento de dominio que utiliza comprende:
resp: valor booleano que indica si el humano cantó algo y tiene que contestar o si puede cantar.
ultimo: el último canto hasta el momento en la ronda actual.
El numero de la mano actual (primera, segunda o tercera)
Resultado de las manos anteriores
Cartas jugadas en la mano actual (propias o del oponente)
Clasificación de las cartas en mano.
Posibles cartas del oponente según los puntos cantados en el envido y las cartas que jugó.

Mediante todos estos atributos se conforman las reglas de inferencia que se dividen (a grandes rasgos) de la 
siguiente forma
1) Contestar un canto:
	a) Se determina cual es la mano actual
	b) Se determina el resultado de la mano anterior (para las manos 2 y 3)
	c) En función de la situación en la que se encuentre y teniendo en cuenta los datos correspondientes al dominio
	   se decide que cantar (Si, No, Re-Truco, Vale 4)
2) Si tengo el quiero 

El conocimiento de dominio necesario fue aportado por los integrantes del grupo 

\section{Futuro Trabajo}
%% Esto que estoy poniendo es para justificar los random pero hay que chamuyarlo mas
Este juego es muy plarticular, no existe una f\'ormula ganadora. Depende mucho de la picard\'ia del jugador. Por eso, apesar de estar basado en un sistema experto
con reglas, en muchos cosas como \'ultima opci\'on se agrego un poco de no-determinismo para crear una sensaci\'on  de que no es una maquin\'a est\'atica. 
Esto podr\'ia cambiarse agregando diferentes estilos de juegos, como el mentiroso, el defensivo o el agresivo.   


\begin{thebibliography}{XXX99}

\bibitem[1]{Reglas} \emph{Descripci\'on y Reglas del Truco }. http://es.wikipedia.org/wiki/Truco\_argentino .

\bibitem[2]{Reglas12} Federico Bergero, Xenofon Floros, Joaqu\'in Fern\'andez, Ernesto Kofman, and
ın a Francois E. Cellier. \emph{Simulating Modelica models with a Stand–Alone Quantized State Systems Solver}. 
In 9th International Modelica Conference, 2012. Aceptado.

\bibitem[3]{Reglas12} Modelica Association. \emph{Modelica 
 - A Unified Object-Oriented Language for Physical Systems Modeling. Language Specification}. 
February 2, 2005.



\bibitem[4]{ModelicaSite} \emph{Sitio de Modelica C Compiler en SourceForge}. http://sourceforge.net/projects/modelicacc/ . 



\end{thebibliography}
\end{document}

